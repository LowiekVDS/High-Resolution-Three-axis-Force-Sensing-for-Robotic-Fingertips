In this chapter the design of the fingertip will be described. We start with the electronics, followed by the mechanical design and the firmware used to read the sensor values.

\section{Electronics}

The electronics of the fingertip are designed to be compact and scaleable. The design supports up to 256 taxels can be read out per I2C-bus. In the last section a discussion will be held 
regarding the scaleability of the sensor on both the electronics and the firmware parts, followed by some guidelines that can be used whilst designing larger sensor arrays.

\subsection{Taxel sensor: MLX9093}

- What is the sensor?
- Conversion time
- Used settings in my case


\subsection{Communication}
- I2c vs SPI

\subsection{PCB Schematic}



\subsection{PCB Layout}

PCB has 4 layers

The sensors are grouped in groups of 4

All connected to a multiplexer.

The PCB was designed in EasyEDA


\subsection{I2C multiplexing}

Because the chosen taxel sensor (MLX9093) can only be configured to have one of four I2C addresses, an I2C multiplexer is required from the moment an array of at least four sensors 
is used on a single I2C-bus. The multiplexer used is the commonly available TCA9548A IC (Integrated Circuit) from Texas Instruments. 
This IC has 8 I2C independant child channels and one parent I2C channel.
The IC itself can be configured to have one of 8 I2C addresses, which can be set by connecting the A0, A1 and A2 pins to either VCC or GND.
In order to configure the IC, one must write a byte to the IC address on the master bus. Every bit corresponds with a child channel, and setting a bit to 1 enables the corresponding child channel.
For example, when someone needs to read out the sensors on the fourth child channel, one must write the byte 00001000 to the IC address.

\subsection{Microcontroller}

The 


\subsection{Scaleability}

